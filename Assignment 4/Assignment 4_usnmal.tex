%% sample template file for a MSc Thesis
%% The default is with two sided setup:
\documentclass{USN-rapport}

% The following command removes the chapter names form the header
% (comment/remove) if you prefer to have them:
\pagestyle{plain}

% --- Bibliography setup ---
%%% default is the "ieee" style
\usepackage[style=ieee, sorting=none]{biblatex}
%%% If you want to use "author-year" style
%%% where `\cite{Foo2011}` generates "Foo et al. (2011)"
%%% and   `\parencite{Foo2011}` generates "(Foo et al. 2011)"
%%% then comment the line above and use
%\usepackage[style=authoryear]{biblatex}
%%% or
%%% if you want to use "alphabetic" style then use
%%% where `cite[Foo2011]` generates "[Foo11]"
%%% then comment the line above and use
%\usepackage[style=alphabetic]{biblatex}
%%% instead.
%% load the bib file:
\addbibresource{thesis.bib}

\usepackage{lipsum} % just for providing fill text used in this template

% --- general setup ---
%% Please fill in the following parameters:
\newcommand{\mysubject}{
	IIA2017
}

\newcommand{\mytitle}{%
%% title:
Title of rapport
}

\newcommand{\myauthor}{%
%% author(thesis) or group code (project):
Torstein Solheim Ølberg, 263054
}

\newcommand{\mysubtitle}{%
	%% master programme (for thesis only)
	%% uncomment the appropriate one:
	%Electrical Power Engineering
	%Energy and Environmental Technology
	%Industrial IT and Automation
	%Process Technology
}

\newcommand{\mykeywords}{%
	%% keywords (for thesis only):
	%s<keyword one, keyword two, \ldots>
}

\newcommand{\myparticipants}{
	%% group participants (for project only)
	<First participant>\\
	<Second participant>\\
	<Third participant>\\
	<Fourth participant>
}

\newcommand{\supervisor}{%
	%% supervisor:
	<Supervisor's Name>}

\begin{document}
	
	% --- title page setup ---
	\USNtitlepage
	

\begin{document}

% --- title page setup ---
\USNtitlepage


\chapter*{Preface}
\label{ch:preface}
\addcontentsline{toc}{chapter}{Preface}
\lipsum[1-3]
\bigskip
Porsgrunn, \today

\myauthor %% for thesis
%\myparticipants %% for project


%% table of contents
\tableofcontents
\addcontentsline{toc}{chapter}{\contentsname}

\listoffigures % out-comment if unwanted
\addcontentsline{toc}{section}{\listfigurename}

\listoftables  % out-comment if unwanted
\addcontentsline{toc}{section}{\listtablename}

\chapter*{Nomenclature}
\label{sec:nomenclature}
bla

\chapter{Introduction}
\label{ch:intro}
\lipsum[4]

\section{Background}
\label{sec:background}

\chapter{Theory}
\label{ch:theory}

\section{Maxwell's Equations}
\label{sec:maxwell}


\section{Mathematical model}
\label{sec:mathmodel}

\lipsum[4]

\lipsum{100-150}

% A dummy command that causes all bibliographyentries to be displayed
% even though there were not cited in the document. Used for demonstration
% purposes only in this template file.
~\nocite{*}

\cleardoublepage

% The bibliography should be displayed here...
\printbibliography[heading=bibintoc]
% You rather like to call the bibliography "References"? Then use this instead:
%\printbibliography[heading=bibintoc, title={References}]


\appendix
%\renewcommand{\appendixname}{Paper} %% So we get 'Paper X' displayed instead


\chapter[Short Title of Paper A]{Title of Paper A (probably very long and therefore not good to have in the header)}
\label{appendix-a}


\chapter[Short Title of Paper B]{Title of Paper B}
\label{appendix-b}

\end{document}

